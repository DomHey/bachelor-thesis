\section{Future work}
There could be problems with the registration, especially in recent developed forums.  In some cases registration requests are protected with additional security mechanisms such as `reCaptcha`. These have not been bypassed yet in the previous implementation, which makes a register in these forums impossible. If a registration is needed in this particular forum the work of Google can be used for developing an algorithm to automatically solve `reCaptchas`\cite{goodfellow2013multi}.
The properly solved `reCaptcha`, along with the other valid register parameters, then allow an automatic registration. If the registration process still does not work, one user account can manually created, verified and used for a login.\\
In some forums the provided e-mail address from 10minutemail~\footnote{https://10minutemail.net/ checked: 10.06.2015} cannot be used for registering. This happens if too many spam accounts have been registered in the forum with an e-mail from this website. To solve the problem , an account must be created manually or the program must be modified so that it registers with a different email provider. The program must still have the possibility to confirm the link in the activation e-mail when using a new provider. If additional parameters are inserted via Javascript during the login or registration process of the website, these will not be handled by the program yet.\\
n a test phase, database sizes should be calculated in order to check whether the formula correctly calculates the database size. if this is not the case, a different weighting of keywords after the Zipf's law\cite{leopold2002zipfsche} can be implemented\cite{jiang2009selectivity}.
Attention should be paid to the general terms and conditions of each forum.
\textbf{If a technical browse of the website is prohibited, this program should not be used!}

\section{Conclusion}
Among the 100 tested internet forums a successful registration could be observed by 76\% of the copied registration pages of the forum. A valid login request could generated by 82\% of the websites. An user account was created manually for 6 web pages but a successful login made with the just generated account. At 80\% of the sites the search form could be identified and a search query was sent correctly. Therefore it should be possible to programatically browse at least 63\% (76\% register x 82\% login) of all internet forums. In order to to make a statement on how relevant this forum is for selling a company product, the demand for a company product expressed in a post is saved. This demand is a numeric value provided by the classification service from Berger und Hennig\cite{n2o} when analyzing a post in order to determine the product. For each extracted post that can be assigned to a company product, this value is added and finally divided by the number of product posts found. This results in the following distribution:

\begin{table}[h!]
\centering
\begin{tabular}{ | p{3cm} | l |}
\hline
\textbf{Produkt} & \textbf{Kaufwunsch}\\ \hline
CRM & 79\% \\ \hline
HCM & 52\% \\ \hline
ECOM & 54\% \\ \hline
LVM & 57\% \\ \hline
\end{tabular}
\caption{Errechneter Kaufbedarf f�r jede Produktkategorie}
\end{table}

This means that almost 80\% of CRM posts in the test database with 22120 posts express a purchase demand. Accordingly, 1816 posts express a strong demand in a CRM product. A statistic is then generated and displayed to the salesman of a company. He can decide whether this forum seems suitable for selling his product. The ratio of product specific contributions to overall database size should not fall below 5\% since then the forum deals with other topics than the product to sell. Furthermore, at least 30\% of the classified posts should express a demand in buying, otherwise , although product specific posts can be found, because the time for finding a post with a demand takes to long and would not be worth the waiting. These numbers need to be evaluated with real salesman in real internet forums, but provide a good guideline for a forums' relevance.
