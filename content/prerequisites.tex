\section{Vorraussetzungen}
Um eine Aussage darüber treffen zu können, wie relevant ein Forum für ein Unternehmen ist, muss dieses Forum durchsuchbar gemacht werden.
Vorraussetzung für ein programmatisches durchsuchen eines Forums ist ein erfolgreicher Login in dem Forum, entweder mit einem vorhandenen Nutzeraccount oder durch eine automatische Registrierung und dann einem Login mit den gerade erstellten Daten.
Der Login in das Forum ist nötig um den geschützten Inhalt des Forums für das Programm sichtbar zu machen. Deshalb wird in diesem Kapitel die Theorie für ein erfolgreiches Registrieren, Einloggen und Durchsuchen eines Forums beschrieben.

%%%%%%%%%%%%%%%%%%%%%%%%%%%%%%%%%%%%%%%%%%%%%%%%%%%%%%%%%%%%%%%%%%%%%%%%%%
%%%%%%%%%%%%%                                Register                                       %%%%%%%%%%%%%%%%%%%%%%%%%%%%%%%%%%%%%%
%%%%%%%%%%%%%%%%%%%%%%%%%%%%%%%%%%%%%%%%%%%%%%%%%%%%%%%%%%%%%%%%%%%%%%%%%
\subsection {Automatisches Registrieren in Foren}
Um an den Inhalt eines Forums zu gelangen muss in den meisten Fällen zunächst ein Nutzerprofil in dem jeweiligen Forum angelegt werden.Anschließend kann sich mit dem Profil eingeloggt werden und der geschützte Inhalt des Forums, als Nutzer einsehen werden. \\
Im folgenden werden die Schritte erläutert, die ein automatisiertes Registrieren auf Webseiten versucht zu ermöglichen.
\subsubsection{Ermittlung der relevanten Formulare in HTML-Webseiten}
Auf der entsprechenden Registrierungsseite des jeweiligen Forums wird eine Analyse auf dem HTML-Quellcode der Seite durchlaufen.
Diese sucht sich aus der Webseite jeden Form-Tag heraus.Es wird ein nummerischen Wert berechnet, mit welcher Wahrscheinlichkeit das gerade gefundene Form-Element das Element ist, das die Input-Elemente für den späteren Registrierungsaufruf enthalten. \\
Es werden Punkte vergeben, wenn in den Attributen des Form-Elements sich bestimmte Schlagworte wie `Signup`, `User`, `Register` etc. wiederfinden. Außerdem muss es sich bei dem Formular um ein POST-Formular handeln, da es bei Absenden des Registrierungsversuches die eingegebenen Informationen wie Nutzername, Passwort und Email im POST-Body mit übermitteln muss.
Deshalb kann man sich bei den meisten Webseiten darauf verlassen, dass wenn Eingabefelder für das Registrieren irgendwo vorhanden sind, diese sich in einem solchen POST-Formular befinden.\\
Ist das Registrierungsformular gefunden, werden die darin vorhandenen Input-Felder analysiert und gewichtet, welches Feld für den Nutzernamen, Passwort und Emailadresse am wahrscheinlichsten ist.
Diese Klassifizierung erfolgt ähnlich wie bei den POST-Formularen der HTML-Webseiten anhand bestimmter Schlagworte.
So sind Input-Felder die in ihren Attributen Wörter wie `Login`, `Username` oder `User` enthalten, wahrscheinlich die Felder in den der Nutzername eingetragen werden soll. Demnach wird für jedes Vorkommen eines dieser Wörter dem Username-Score ein fester Wert aufaddiert. Sollten hingegen in den Attributen Wörter wie `Password` oder `Email`, dem Passwort- oder Emailscore des gerade betrachteten Inputfeldes,wird ein Wert hinzuaddiert.
Zum Schluss wird die Klassifizierung des Inputfeldes anhand des höchsten Wertes von Email -, Passwort- und Nutzernamen-Wert vorgenommen. \\
Sind alle Input-Felder klassifiziert werden die Checkboxen, sofern sich welche in dem Formular befinden, angeschaut. Diese sind meist dazu da, um die Foren-AGB zu akzeptieren, angemeldet zu bleiben oder den Newsletter des jeweiligen Forums zu akzeptieren.
Es werden getrost alle Checkboxen angehakt. Der Newsletter, sofern die Checkbox vorhanden sein sollte, ist nicht störend, da eine Spam-Email-Adresse für das Registrieen benutzt wird.\\
Zum Schluss werden sich alle Inputfelder vorgenommen die `versteckt` sind, also das Attribut `hidden`, im HTML-Quellcode enthalten.
Diese Felder sind nicht dazu da, damit der Nutzer dort Informationen einträgt, sonder lediglich um bei der Registrierungsanfrage zusätzliche Daten mitzusenden. Daher werden die Key-Value Paare des Input-Feldes zusammen mit den vorher gesammelten Daten an das Absenden der Registrierungsanfrage übergeben.
\subsubsection{Absenden der Registrierungsanfrage}
Aus dem vorherigen Schritt sind nun alle relevanten Daten für den Registrierungsprozess erfasst worden. Nun wird ein zufälliger Nutzername generiert. Dazu wird der externe Service randomuser.me\footnote{http://api.randomuser.me/} benutzt. Ein Aufruf dieser API liefert JSON Information zu einem generierten Nutzer zurück, samt Nutzername. Dieser wird an die Stellen des POST-Formulares als Value des Keys eingetragen, der den Nutzernamen verlangt.\\
Im folgenden wird ein Passwort generiert, welches jeden Sicherheitskriterien genügt. Das heißt Groß- und Kleinbuchstaben, Sonderzeichen und mindestens 10 Zeichen lang. Dieses wird an jede Passwort-Value Stelle eingefügt.\\
Bei der erstmaligen Registrierung in einem Forum wir oft eine Email von dem Forum versandt um sicherzustellen, ob der Anmelder eine valide Email bei der Registrierung angegeben hat oder nicht. Das dient auch dazu sich vor sich vor einer programmatischen Registrierung zu schützen. Um immer eine valide Emailadresse angeben zu können,wird ein externer Service benutzt, 10minutemail.net \footnote{https://10minutemail.net/}, der für 10 Minuten eine valide Emailadresse zur Verfügung stellt. Diese Emailadresse wir nun als Value bei den Email-Input-Feldern eingegeben.\\
Nun sind alle Key-Value Paare im Post-Body der Registrierungsanfrage valide und die Anfrage kann abgesendet werden.
Die URL an die der Registrierungsversuch gesendet wird, setzt sich aus der Top-Level-Domain der Webseite und der `action`- value des POST- Formulares, sowie dem generierten POST-Body, zusammen.
\subsubsection{Bestätigen des Registrierungslinks in der Registrierungsemail}
Nachdem die Registrierungsaufvorderung in dem Forum eingeht, wird eine Bestätigungsmail an die Emailadresse gesendet.
Um diese Email nun abrufen zu können, wird alle 5 Sekunden die Seite neu geladen, damit eine etwaige neue Email angezeigt wird. Da die Email-Webseite keinen Login erfordert, wird die Zuordnung zwischen Nutzer und zugehörigen Emails über Cookies gelöst. Diese Cookies werden bei dem ersten Besuchen der Webseite, wenn die Email erstellt wird, im Response-header an den Nutzer mit übergeben. Wenn bei dem programmatischen Neuladen der Webseite diese Cookies mitgesendet werden, können Emails, die an die Adresse gesendet werden, abgerufen werden. Bei fehlen der Cookies würde immer eine neue Emailadresse angelegt.\\
Wird nun die Email des Forums in dem Posteingang gefunden, werden alle Hyperlinks aus dem Emailinhalt extrahiert und einmal mit einem GET-Request aufgerufen. Dieses Versichert, dass das angelegte Nutzerprofil in dem Forum aktiviert und validiert wird und zum einloggen freigeschaltet wird.

%%%%%%%%%%%%%%%%%%%%%%%%%%%%%%%%%%%%%%%%%%%%%%%%%%%%%%%%%%%%%%%%%%%%%%%%
%%%%%%%%%%%%%                                LOGIN                                       %%%%%%%%%%%%%%%%%%%%%%%%%%%%%%%%%%%%%
%%%%%%%%%%%%%%%%%%%%%%%%%%%%%%%%%%%%%%%%%%%%%%%%%%%%%%%%%%%%%%%%%%%%%%%%

\subsection {Automatisches Einloggen in Foren}
Das Login ist wichtig um die Cookies der Webseite zu erhalten, die dann bei jeder neuen Anfrage an die Webseite wieder mitgesendet werden müssen um zu validieren das die Anfrage von einem eingeloggten Nutzer stammt. Damit wird der geschützte Bereich des Forums zugänglich.
\subsubsection{Ermittlung der relevanten Formulare in HTML-Webseiten}
Die Ermittlung der relevanten Login-Form sowie den Input-Feldern erfolgt nach dem Prinzip der Suche nach den relevanten Registrierungsformularen. Es werden zuerst alle POST-Formulare des HTML-Quellcodes extrahiert und bewertet. Der Unterschied besteht darin, dass die Anzahl der Input-Felder in der Form theoretisch zwischen 2 und 3 liegen müssten, da als Maximum Email, Nutzername und Passwort angegeben werden müssen. Das Minimum ist hingegen Nutzername oder Email und das Passwort. Alle Formulare mit mehr Eingabemöglichkeiten sind höchst warscheinlich nicht das Loginformular.
Sollte es mehr als ein Formular geben, das zwischen 2 und 3 Input-Feldern besitzt, müssen die Input-Felder genauer in Betracht gezogen werden. \\
Enthalten diese in ihren Attributen differenzierte Schlagworte wie `login`, `username`, `passwort` oder `email` werden für jedes Auftreten dieser Schlagworte Punkte auf den Punktewert dieser Form aufgerechnet. Weiterhin werden die Input-Typs der Input-Formen analysiert. Bestehen sie aus den Kombinationen, Input-Type = 'text' und der Input-Name = Einer Kombination aus den Worten `login`, `username` oder `email`, werden weitere Punkte zu dem Score hinzugerechnet.
\[Inputtype = `text` \wedge Inputname = [`login` \vee `username` \vee `email`]^*\]

 Selbiges gilt für eine Kombination aus Input-Type = `password` und Input-Name = `password`.
 \[Inputtype = `password` \wedge Inputname = `password`\]
 
 
Sollte die Form genau ein Input-Feld mit dem Type = `password`, ein Input-Feld mit dem Type = `text ` und eine Checkbox haben ist diese Form am wahrscheinlichsten die Login-Form. Sie bestünde aus einem Feld für den Usernamen oder Email, einem Passwortfeld und einer Checkbox, ob ein angemeldet bleiben erwünscht ist.\\
Sind alle Formen auf der Seite analysiert ist die Form mit dem höchsten Score, die Form die für den Login-Request benutzt wird. Eine Klassifizierung der Input-Felder, ob sie den Usernamen, Email oder Passwort übermitteln, wurde im vorherigen Schritt schon angelegt, da sich daraus der Form-Score berechnet. Zum Schluss werden noch alle zusätzlichen Input-Felder, die die Eigenschaft `visibility=hidden` haben, als Key-Value Paar zusammen mit den Namen der Email,Passwort und Nutzername Feldern und deren Klassifizierung im JSON-Format gespeichert.
\subsubsection{Absenden des Loginrequests}
Der Request der an den Server abgesendet wird ist ein POST-Request, da er die Formulardaten mit an den Server übermitteln muss.
An die Value Stelle der klassifizierten Input-Namen werden nun die entsprechenden Daten eingefüllt und an den Server gesendet.
Sollte dieser Request scheitern, werden die restlichen Formen, die sich aus dem HTML-Quellcode extrahieren ließen als mögliche Login-Formen klassifiziert ausprobiert.
\subsubsection{Handhabung von Cookies und Redirects}
Sollte in dem Response Header des Servers bei einer Loginanfrage ein Stauscode 200 gesendet werden, war der Loginversuch erfolgreich. Da das HTTP-Protokoll keine Zustände speichern kann, muss dem Server bei jeder neuen Anfrage mitgeteilt werden, dass die Anfrage nach einem erfolgreichen Loginversuch unternommen wurde. Andernfalls würde der Server den Zugriff auf die Daten, die nach einem erfolgreichen Login erreicht werden könnten, sperren.\\
Um dieses zu umgehen werden die Cookies aus dem Response-Header der erfolgreichen Loginanfrage gespeichert und bei jeder neuen Anfrage an den Server mitgesendet.\\
Sollte in dem Resopnse-Header ein Feld `Location` existieren bedeutet das, dass es einen Redirect nach dem Login geben würde, bei dem der Browser eine andere URL laden würde. Dieser Redirect ist für den nachfolgenden Schritt, das suchen der Suchbar in Foren wichtig. Die Suchbar wird oft auf der Startseite des Forums angezeigt und der Redirect führt genau dort hin. Deshalb wird die URL, die sich in dem Location-Header durch einen GET-Request geladen und für die weitere Verarbeitung genutzt, um sicherzustellen, dass es auch eine Such-Form in dem zu analysierenden HTML-Quellcode gibt.

%%%%%%%%%%%%%%%%%%%%%%%%%%%%%%%%%%%%%%%%%%%%%%%%%%%%%%%%%%%%%%%%%%%%%%%%
%%%%%%%%%%%%%                                SUCHEN                                     %%%%%%%%%%%%%%%%%%%%%%%%%%%%%%%%%%%%%
%%%%%%%%%%%%%%%%%%%%%%%%%%%%%%%%%%%%%%%%%%%%%%%%%%%%%%%%%%%%%%%%%%%%%%%%

\subsection{Ermitteln der Größe der Forendatenbank}
Die Ermittelung der Datenbankgröße wird bei dem späteren treffen der Aussage über die Forenrelevanz für ein Produkt wichtig.
Es ist wichtig die annähernde Anzahl der Posts in dem Forum zu kennen, um bei einer Langzeitanalyse festzustellen, wie schnell das Forum wächst, bzw. wie viele Einträge pro Tag hinzu kommen. Daraus kann ermittelt werden, mit welcher Wahrscheinlichkeit neue Beiträge zu einem Verkaufsprodukt mit welcher Frequenz eingestellt werden.\\
\subsubsection{Ermitteln des Suchfeldes}
Um die Datenbank analysieren zu können muss zunächst das Suchfeld, welches es in den meisten Foren gibt, um die vorhandenen Posts zu durchsuchen, gefunden werden.
Dazu wird das HTML das nach dem erfolgreichen Einloggen auf weitere Form-Felder durchsucht. Die gefundenen Felder werden klassifiziert indem sich die Namen-Attribute der Input-Felder dieser Form angeschaut werden. Sollte dieser Name aus nur einem Buchstaben bestehen, vorzugsweise ein `q`, da das bei den meisten Webseiten für den Paramater query steht, werden Punkte vergeben.
Danach werden alle Attribute der Input-Felder der Form durchsucht. Sollten in den Attributen die Zeichenketten `search` oder `suche` vorkommen werden weitere Punkte vergeben.\\
Zum Schluss ist die Form mit dem höchsten Punktwert, die Form die am wahrscheinlichsten die Suchbar als Input-Feld enthält.
Nun wird der Pfad der Suche aus der Form extrahiert, indem das `action` Attribut der Form ausgelesen wird. Dieser Wert wird an die Toplevel Domain angefügt und stellt die Suchurl dar. Der Queryparameter wird aus dem Name-Attribut des Input-Feldes extrahiert und an die Suchurl angehangen. Sollte zum Beispiel das Name-Attribut des Input-Feldes ein `q` sein würde sich die vorläufige Suchurl so zusammensetzen: `Toplevel Url` + `?q=`. An den Queryparameter q werden im folgenden die einzelnen Suchworte, die zur Ermittelung der Datenbankgröße beitragen angehängt und an den Server abgesendet.\\
Diese Wörter, die die Datenbank bestimmen sollen werden aus einer Datei gelesen, die die am häufigsten benutzen Wörter in einer Sprache sortiert auflistet. Aus diesen Wörtern werden zufällig 500 Wörter gezogen ,in die Url eingefügt und abgesendet.\\
\subsubsection{Berechnung der Datenbankgröße}
Nach jedem abgesendeten Request wird der zurückgegebene Inhalt analysiert. Dieser enthält die Posts, die von der 
forumseigenen Suchmaschine zurückgegeben werden, wenn man nach einem spezifischen Wort sucht. In dem Inhalt werden nun alle Hyperlinks gesucht und gespeichert. Diese Prozedur wird 500 mal wiederholt. Zum Schluss werden die Gesamtzahl an gewonnenden Links berechnet, sowie die Anzahl der Links, die bei unterschiedlichen Suchwörtern identisch sind. Links die in mehr als 50\% der Gesamtanfragen vorkommen werden ignoriert, da es sich meist hierbei dann um statische Links die auf der Suchseite befinden handeln. Diese Links zeigen dann nicht auf Posts die aus der Suchanfrage generiert werden, sondern auf statische Elemente der HTML-Webseite wie Navigationsbar oder ähnliches. Diese würden dann die Berechnung der Datenbankgröße verfälschen und werden deshalb ignoriert.
Die finale Größe der Datenbank lässt sich mit Hilfe der Formel \[\frac{u}{1-OR^{-1.1}}\] \cite{lu2008efficient} berechnen. Hierbei ist u die Anzahl der einzigartigen Dokumente und OR das Verhältnis von allen Dokumenten zu doppelten. Das Verhältnis von einzigartigen zu doppelten Dokumenten bestimmt die Datenbankgröße.\\
Dieser ganze Vorgang wird drei mal wiederholt und aus den drei resultierenden Datenbankgrößen das arithmetische Mittel gebildet. Diese Zahl bestimmt, wie viele Posts sich insgesamt in der Forendatenbank befinden.

%%%%%%%%%%%%%%%%%%%%%%%%%%%%%%%%%%%%%%%%%%%%%%%%%%%%%%%%%%%%%%%%%%%%%%%%
%%%%%%%%%%%%%                              FORENRELEVANZ                                 %%%%%%%%%%%%%%%%%%%%%%%%%%%%%%%%%%
%%%%%%%%%%%%%%%%%%%%%%%%%%%%%%%%%%%%%%%%%%%%%%%%%%%%%%%%%%%%%%%%%%%%%%%%


\subsection{Forenrelevanz}
Um eine gezielte Aussage darüber treffen zu können, ob das gerade analysierte Forum überhaupt relevant für die Verkaufsstrategie eines Unternehmens ist, muss neben der Gesamtgröße des Forums auch die Anzahl und Qualität der Posts für Verkaufsprodukte analysiert werden. Das Verhältnis zwischen Gesamtforengröße, Postanzahl zu jedem einzelnen Unternehmensprodukt und die Qualität des Postes, gemessen an dem ausgedrückten Verlangen etwas zu kaufen, geben Aufschluss darüber, ob das Forum für eine Firma als wichtig erachtet werden kann.
\subsubsection{Generierung der Produktschlagworte}
Die Ermittelung der Gesamtanzahl der Posts die zu einem Firmenprodukt zugeordnet werden, werden in der gleichen Art und Weise bestimmt wie die Gesamtgröße der Forendatenbank.\\
Zu diesem Zweck werden Dokumente der Firmen analysiert, die eindeutig ein Produkt beschreiben. Dieses können Broschüren, Webseiten oder PDF-Datein sein. Diese Dateien können von der jeweiligen Firma hochgeladen werden. Aus diesen Texten werden die häufigsten Wörter nach Tf-idf-Maß extrahiert.Aus der generierten Wortliste werden die häufigsten Wörter, in der jeweilig genutzten Sprache, entfernt. Diese entfernten Wörter sind Füllwörter die, wenn sie gesucht werden, beliebige Posts als Suchergebnis liefern würden, deren Inhalt nichts mit dem eigentlich gesuchten Produkt gemein haben.\\
Ist die Wortliste generiert werden diese Worte in der Suchmaske des Forums gesucht und wiederum die Links in der Antwort gespeichert und gezählt. Hinzu kommt, das jeder gefundene Link zu einem Post durch einen weiteren Get-Request an diese Seite aufgerufen und heruntergeladen wird. Diese heruntergeladene Seite gibt den selben Zustand wieder, als ob ein Mensch mit dem Browser sich eingeloggt und diesen Post aufgerufen hätte. Aus dem Dokument werden alle HTML-Tags entfernt. Im Anschluss wird der Postinhalt extrahiert und an einen Service geschickt, der den Inhalt genauer analysiert. Der Postinhalt ist der erste zusammenhängende Text der 40 oder mehr Zeichen enthält \footnote{http://blog.viermalvier.at/die-richtige-laenge-von-social-media-postings-mit-infografik/ checked: 25.06.2015}, da dieses die optimale Anzahl an Buchstaben in einem Social Media Post ist.
\subsubsection{Analyse der Forenposts}
Dieser Service bewertet den Inhalt auf Grund der vorher hoch geladenen Firmenbroschüren. Diese Broschüren beschreiben genau ein Produkt genauer. Die Gesamtheit aller Broschüren die ein Produkt klassifizieren, bilden Produktcluster. Jeder Eintrag verweist mit einem Distanzvektor auf die nächsten ihm ähnlichen Broschüreneinträge. Soll nun ein gerade gewonnener Postinhalt klassifiziert werden, werden die ihm ähnlichsten Broschüreneinträge berechnet. Über knn wird nun bestimmt, zu welchem Produkt der Text des Postes am ähnlichsten ist. Ist der Post einem bestimmten Produkt zugeordnet wird der Kaufbedarf der in dem Post zum Ausdruck gebracht wird berechnet. Dieses geschieht mit einer Support Vektor Maschine, die im Rahmen des Projektes Noise to Opportunity entwickelt wurde \cite{n2o}.\\
Wird ein Post einem Produkt zugeordnet und drückt dieser Post einen Kaufwillen aus, wird er für die weitere Berechnung der Relevanz des Forums in Betracht gezogen.
\subsubsection{Ermittelung der Datenbankgröße je Firmenprodukt}
Die Links der Posts, die sowohl einem Produkt als auch einen Kaufwillen ausdrücken werden gespeichert. Die Ermittlung der Datenbankgröße je Firmenprodukt erfolgt analog zu der Ermittlung der Gesamtdatenbankgröße des Forums. Die Formel \[\frac{u}{1-OR^{-1.1}}\] \cite{lu2008efficient} liefert auch hier die Größe der Datenbankeinträge abgeschätzt aus dem Verhältnis der einzigartigen und sich überschneidenden Dokumente zurück.\\
Sollten mehr als 30\% der analysierten Postinhalte zu einem Firmenprodukt einen Verkaufsbedarf ausdrücken, kann dieses Forum für die Firma als wichtig erachtet werden.