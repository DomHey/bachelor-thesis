\section {Automatisches Registrieren in Foren}
Um an den Inhalt einesd Forums zu gelangen muss in den meisten Fällen zunächst ein Nutzerprofil in dem jeweiligen Forum angelegt werden.Anschließend kann sich mit dem Profil eingeloggt werden und der geschützte Inhalt des Forums, als Nutzer einsehen werden. \\
Im folgenden werden die Schritte erläutert, die ein automatisiertes Registrieren auf Webseiten versucht zu ermöglichen.
\subsection{Ermittlung der relevanten Formulare in HTML-Webseiten}
Auf der entsprechenden Registrierungsseite des jeweiligen Forums wird eine Analyse auf dem HTML-Quellcode der Seite durchlaufen.
Diese sucht sich aus der Webseite jeden Form-Tag heraus.Es wird ein nummerischen Wert berechnet, mit welcher Wahrscheinlichkeit das gerade gefundene Form-Element das Element ist, das die Input-Elemente für den späteren Registrierungsaufruf enthalten. \\
Es werden Punkte vergeben, wenn in den Attributen des Form-Elements sich bestimmte Schlagworte wie `Signup`, `User`, `Register` etc. wiederfinden. Außerdem muss es sich bei dem Formular um ein POST-Formular handeln, da es bei Absenden des Registrierungsversuches die eingegebenen Informationen wie Nutzername, Passwort und Email im POST-Body mit übermitteln muss.
Deshalb kann man sich bei den meisten Webseiten darauf verlassen, dass wenn Eingabefelder für das Registrieren irgendwo vorhanden sind, diese sich in einem solchen POST-Formular befiden.\\
Ist das Registrierungsformular gefunden, werden die darin vorhandenen Input-Felder analysiert und gewichtet, welches Feld für den Nutzernamen, Passwort und Emailadresse am wahrscheinlichsten ist.
Diese Klassifizierung erfolgt ähnlich wie bei den POST-Formularen der HTML-Webseiten anhand bestimmter Schlagworte.
So sind Input-Felder die in ihren Attributen Wörter wie `Login`, `Username` oder `User` enthalten, wahrscheinlich die Felder in den der Nutzername eingetragen werden soll. Demnach wird für jedes Vorkommen eines dieser Wörter dem Username-Score ein fester Wert aufaddiert. Sollten hingegen in den Attributen Wörter wie `Password` oder `Email`, dem Passwort- oder Emailscore des gerade betrachteten Input-Feldes,wird ein Wert hinzuaddiert.
Zum Schluss wird die Klassifizierung des Input-Feldes anhand des höchsten Wertes von Email -, Passwort- und Nutzernamen-Wert vorgenommen. \\
Sind alle Input-Felder klassifiziert werden die Checkboxen, sofern sich welche in dem Formular befinden, angeschaut. Diese sind meist dazu da, um die Foren-AGB zu akzeptieren, angemeldet zu bleiben oder den Newsletter des jeweiligen Forums zu aktzeptieren.
Es werden getrost alle Checkboxen angehakt. Der Newsletter, sofern die Checkbox vorhanden sein sollte, ist nicht störent, da eine Spam-Email-Adresse für das Registrieen benutzt wird.\\
Zum Schluss werden sich alle Input-felder vorgenommen die `versteckt` sind, also das Attribut `hidden`, im HTML-Quellcode haben.
Diese Felder sind nicht dazu da, damit der Nutzer dort Informationen einträgt, sonder lediglich um bei der Registrierungsanfrage zusätzliche Daten mitzusenden. Daher werden die Key-Value Paare des Input-Feldes zusammen mit den vorher gesammelten Daten an das Absenden der Registrierungsanfrage übergeben.
\subsection{Absenden der Registrierungsanfrage}
Aus dem vorherigen Schritt sind nun alle relevanten Daten für den Registrierungsprozess erfasst worden. Nun wird ein zufälliger Nutzername generiert. Dazu wird der externe Service `randomuser.me` [!!!!!!!!] benutzt. Ein Aufruf dieser API liefert JSON Information zu einem generierten Nutzer zurück, samt Nutzername. Dieser wird an die Stellen des POST-Formulares als Value des Keys eingetragen, der den Nutzernamen verlangt.\\
Im folgenden wird ein Passwort generiert, welches jeden Sicherheitskriterien genügt. Das heist Groß- und Kleinbuchstaben, Sonderzeichen und mindestens 10 Zeichen lang. Dieses wird an jede Passwort-Value Stelle eingefügt.\\
Bei der erstmaligen Registrierung in einem Forum wir oft eine Email von dem Forum versandt um sicherzustellen, ob der Anmelder eine valide Email bei der Registrierung angegeben hat oder nicht. Das dient auch dazu sich vor sich vor einer programmatischen Registrierung zu schützen. Um immer eine valide Emailadresse angeben zu können,wird ein externer Service benutzt, der für 10 Minuten eine valide Emailadresse zur Verfügung stellt. Diese Emailadresse  wir nun als Value bei den Email-Input-Feldern eingegeben.\\
Nun sind alle Key-Value Paare im Post-Body der Registrierungsanfrage valide und die Anfrage kann abgesendet werden.
Die URL an die der Registrierungsversuch gesendet wird, setzt sich auss der Top-Level-Domain der Webseite und der `action`- value des Post-Formulares, sowie dem generierten POST-Body, zusammen.
\subsection{Bestätigen des Registrierungslinks in der Registrierungsemail}
Nachdem die Registrierungsaufvorderung in dem Forum eingeht, wird eine Bestätigungsemail an die Emailadresse gesendet.
Um diese Email nun abrufen zu können, wird alle 5 Sekunden die Seite neu geladen, damit eine etwaige neue Email angezeigt wird. Da die Email-Webseite keinen Login erfordert, wird die Zuordung zwischen Nutzer und zugehörigen Emails über Cookies gelköst. Diese Cookies werden bei dem ersten Besuchen der Webseite, wenn die Email erstellt wird, im Response-header anden Nutzer mit übergeben. Wenn bei dem programmatischen Neuladen der Webseite diese Cookies mitgesendet werden, können Emails, die an die Adresse gesendet werden, abgerufen werden. Bei fehlen der Cookies würde immer eine neue Emailadresse angelegt.\\
Wird nun die Email des Forums in dem Posteingang gefunden, werden alle Hyperlinks aus dem Emailinhalt extrahiert und einmal mit einem GET-Request aufgerufen. Dieses Versichert, dass das angelegte Nutzerprofil in dem Forum aktiviert und validiert wird und zum einloggen freigeschaltet wird.
\newpage