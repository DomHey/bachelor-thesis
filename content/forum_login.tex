\section {Automatisches Einloggen in Foren}
Das Login ist wichtig um die Cookies der Webseite zu erhalten, die dann bei jeder neuen Anfrage an die Webseite wieder mitgesendet werden müssen um zu validieren das die Anfrage von einem eingeloggten Nutzer stammt. Damit wird der geschützte Bereich des Forums zugänglich.
\subsection{Ermittlung der relevanten Formulare in HTML-Webseiten}
Die Ermittlung der relevanten Login-Form sowie den Input-Feldern erfolgt nach dem Prinzip der Suche nach den relevanten Registrierungsformularen. Es werden zuerst alle POST-Formulare des HTML-Quellcodes extrahiert und bewertet. Der Unterschied besteht darin, dass die Anzahl der Input-Felder in der Form theoretisch zwischen 2 und 3 liegen müssten, da als Maximum Email, Nutzername und Passwort angegeben werden müssen. Das Minimum ist hingegen Nutzername oder Email und das Passwort. Alle Formulare mit mehr Eingabemöglichkeiten sind höchst warscheinlich nicht das Loginformular.
Sollte es mehr als ein Formular geben, das zwischen 2 und 3 Input-Feldern besitzt, müssen die Input-Felder genauer in Betracht gezogen werden. \\
Enthalten diese in ihren Attributen differenzierte Schlagworte wie `login`, `username`, `passwort` oder `email` werden für jedes Auftreten dieser Schlagworte Punkte auf den Punktewert dieser Form aufgerechnet. Weiterhin werden die Input-Typs der Input-Formen analysiert. Bestehen sie aus den Kombinationen, Input-Type = 'text' und der Input-Name = Einer Kombination aus den Worten `login`, `username` oder `email`, werden weitere Punkte zu dem Score hinzugerechnet. Selbiges gilt für eine Kombination aus Input-Type = password und Input-Name = `password`.\\
Sollte die Form genau ein Input-Feld mit dem Type = `password`, ein Input-Feld mit dem Type = `text ` und eine Checkbox haben ist diese Form am wahrscheinlichsten die Login-Form. Sie bestünde aus einem Feld für den Usernamen oder Email, einem Passwortfeld und einer Checkbox, ob ein angemeldet bleiben erwünscht ist.\\
Sind alle Formen auf der Seite analysiert ist die Form mit dem höchsten Score, die Form die für den Login-Request benutzt wird. Eine Klassifizierung der Input-Felder, ob sie den Usernamen, Email oder Passwort übermitteln, wurde im vorherigen Schritt schon angelegt, da sich daraus der Form-Score berechnet. Zum Schluss werden noch alle zusätzlichen Input-Felder, die die Eigenschaft `visibility=hidden` haben, als Key-Value Paar zusammen mit den Namen der Email,Passwort und Nutzername Feldern und deren Klassifizierung im JSON-Format gespeichert.
\subsection{Absenden des Loginrequests}
Der Request der an den Server abgesendet wird ist ein POST-Request, da er die Formulardaten mit an den Server übermitteln muss.
An die Value Stelle der klassifizierten Input-Namen werden nun die entsprechenden Daten eingefüllt und an den Server gesendet.
Sollte dieser Request scheitern, werden die restlichen Formen, die sich aus dem HTML-Quellcode extrahieren ließen als mögliche Login-Formen klassifiziert ausprobiert.
\subsection{Handhabung von Cookies und Redirects}
Sollte in dem Response Header des Servers bei einer Loginanfrage ein Stauscode 200 gesendet werden, war der Loginversuch erfolgreich. Da das HTTP-Protokoll keine Zustände speichern kann, muss dem Server bei jeder neuen Anfrage mitgeteilt werden, dass die Anfrage nach einem erfolgreichen Loginversuch unternommen wurde. Andernfalls würde der Server den Zugriff auf die Daten, die nach einem erfolgreichen Login erreicht werden könnten, sperren.\\
Um dieses zu umgehen werden die Cookies aus dem Response-Header der erfolgreichen Loginanfrage gespeichert und bei jeder neuen Anfrage an den Server mitgesendet.\\
Sollte in dem Resopnse-Header ein Feld `Location` existieren bedeutet das, dass es einen Redirect nach dem Login geben würde, bei dem der Browser eine andere URL laden würde. Dieser Redirect ist für den nachfolgenden Schritt, das suchen der Suchbar in Foren wichtig. Die Suchbar wird oft auf der Startseite des Forums angezeigt und der Redirect führt genau dort hin. Deshalb wird die URL, die sich in dem Location-Header durch einen GET-Request geladen und für die weitere Verarbeitung genutzt, um sicherzustellen, dass es auch eine Such-Form in dem zu analysierenden HTML-Quellcode gibt.
\newpage

