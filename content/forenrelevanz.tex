\section{Forenrelevanz}
Um eine gezielte Aussage darüber treffen zu können, ob das gerade analysierte Forum überhaupt relevant für die Verkaufsstrategie eines Unternehmens ist, muss neben der Gesamtgröße des Forums auch die Anzahl und Qualität der Posts für Verkaufsprodukte analysiert werden. Das Verhältnis zwischen Gesamtforengröße, Postanzahl zu jedem einzelnen Unternehmensprodukt und die Qualität des Postes, gemessen an dem ausgedrückten Verlangen etwas zu kaufen, geben Aufschluss darüber, ob das Forum für eine Firma als wichtig erachtet werden kann.
\subsection{Generierung der Produktschlagworte}
Die Ermittelung der Gesamtanzahl der Posts die zu einem Fimenprodukt zugeordnet werden, werden in der gleichen Art und Weise bestimmt wie die Gesamtgröße der Forendatenbank.\\
Zu diesem Zweck werden Dokumente der Firmen analysiert, die eindeutig ein Produkt beschreiben. Dieses können Broschüren, Webseiten oder PDF-Datein sein. Diese Datein können von der jeweiligen Firma hochgeladen werden. Aus diesen Texten werden die häufigsten Wörter nach Tf-idf-Maß extrahiert.Aus der generierten Wordliste werden die häufigsten Wörter, in der jeweilig genutzen Srache, entfernt. Diese entfernten Wörter sind Füllwörter die, wenn sie gesucht werden, beliebige Posts als Suchergebnis liefern würden, deren Inhalt nichts mit dem eigentlich gesuchten Produkt gemein haben.\\
Ist die Wortliste generiert werden diese Worte in der Suchmaske des Forums gesucht und wiederum die Links in der Antwort gespeichert und gezählt. Hinzu kommt, das jeder gefundene Link zu einem Post durch einen weiteren Get-Request an diese Seite aufgerufen und runtergeladen wird. Diese runtergeladene Seite gibt den selben Zustand wieder, als ob ein Mensch mit dem Browser sich eingeloggt und diesen Post aufgerufen hätte. Aus dem Dokument werden alle HTML-Tags entfernt. Im Anschluss wird der Postinhalt extrahiert und an einen Service geschickt, der den Inhalt genauer analysiert. Der Postinhalt ist der erste zusammenhängende Text der 40 oder mehr Zeichen enthält \footnote{http://blog.viermalvier.at/die-richtige-laenge-von-social-media-postings-mit-infografik/ checked: 25.06.2015}, da dieses die optimale Anzahl an Buchstaben in einem Social Media Post ist.
\subsection{Analyse der Forenposts}
Dieser Service bewertet den Inhalt auf Grund der vorher hochgeladenen Firmenbroschüren. Diese Broschüren beschreiben genau ein Produkt genauer. Die Gesamtheit aller Broschüren die ein Produkt klassifizieren, bilden Produktcluster. Jeder Eintrag verweist mit einem Distanzvektor auf die nächsten ihm ähnlichen Broschüreneinträge. Soll nun ein gerade gewonnener Postinhalt klassifiziert werden, werden die ihm ähnlichsten Broschüreneinträge berechnet. Über knn wird nun bestimmt, zu welchem Produkt der Text des Postes am ähnlichsten ist. Ist der Post einem bestimmten Produkt zugeordnet wird der Kaufbedarf der in dem Post zum Ausdruck gebracht wird berechnet. Dieses geschieht mit einer Support Vektor Maschine, die im Rahmen des Projektes Noise to Opportunity entwickelt wurde \cite{n2o}.\\
Wird ein Post einem Produkt zugeordnet und drückt dieser Post  einen Kaufwillen aus, wird er für die weitere Berechnung der Relevanz des Forums in Betracht gezogen.
\subsection{Ermittelung der Datenbankgröße je Firmenprodukt}
Die Links der Posts, die sowohl einem Produkt als auch einen Kaufwillen ausdrücken werden gespeichert. Die Ermittlung der Datenbankgröße je Firmenprodukt erfolgt analog zu der Ermittlung der Gesamtdatenbankgröße des Forums. Die Formel \(\frac{u}{1-OR^{-1.1}}\) \cite{lu2008efficient} liefert auch hier die Größe der Datenbankeinträge abgeschätzt aus dem Verhältnis der einzigarten und sich überschneidenden Dokumente zurück.\\
Sollten mehr als 30\% der analysierten Postinhalte zu einem Firmenprodukt einen Verkaufsbedarf ausdrücken, kann dieses Forum für die Frima als wichtig erachtet werden.

\newpage