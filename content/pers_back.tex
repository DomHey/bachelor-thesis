\section{Persönlicher Hintergrund}
Im Rahmen eines Bachelorprojektes am Hasso-Plattner-Institut wurde von einer Gruppe von 8 Personen  innerhalb eines Jahres die Softwarelösung `Noise to Opportunity` entwickelt. `Noise to Opportunity` durchsucht soziale Medien und analysiert die gefundenen Beiträge. Werden diese einem Unternehmensprodukt zugeordnet und drücken einen Kaufwunsch aus, werden sie an Salesmen weitergeleitet, die auf den Verkauf dieses Produktes spezialisiert sind. Ich habe mich mit der Datenakquise beschäftigt. Dazu habe ich eine Vielzahl von Webseiten analysiert, um die Registrierungs-, Einlogg- und Suchprozesse zu verstehen und programmtechnisch nachzubauen. Bei dieser Tätigkeit ist aufgefallen, dass die meisten Internetforen eine ähnliche Struktur besitzen. Anfänglich war die Herausforderung, sich mit einem händisch angelegten Nutzeraccount im Forum automatisch anzumelden und das Forum zu durchsuchen. Die Analyse der Webseiten ließ sich wiederholende Prozesse erkennen. So entstand die Idee, sich auch automatisch registrieren zu wollen.
Viele der Registrierungsformen besitzen gleiche Attributnamen, die sie in dem HTML-Quellcode identifizieren. Der Registrierungsprozess folgt dem gleichen Schema und lässt sich demnach genau wie das Einloggen und das Suchen automatisieren. Es entstand die Idee, ein Programm zu entwickeln, das automatisch Internetinhalte hinter POST- und GET-Formularen des Deep- und Private Webs indizieren kann. Wir sahen die Möglichkeit, eine Voraussage darüber zu treffen, wie viele Beiträge zu einem bestimmten Produkt in einem Forum vorhanden sind.
Daraus entwickelte sich die These dieser Arbeit, dass es möglich sei abzuschätzen, wie verkaufsrelevant das Forum für Unternehmensprodukte ist.
\newpage