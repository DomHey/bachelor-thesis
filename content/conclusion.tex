\section{Schlussfolgerung}
Der Anteil der gesamten klassifizierbaren Beiträge zur Gesamtdatenbankgröße sollte mindestens 30 \% betragen, damit schnell verkaufsträchtige Posts gefunden werden. Hingegen, sollte der Anteil von spezifischen Posts zur Gesamtdatenbankgröße nicht unter 10 \% fallen, da sonst der Aufwand und die Zeit, diese Beiträge der Kategorie zu erfassen, nicht im Verhältnis zum Gewinn stehen. So wäre in diesem Fall der Teil der Datenbank für CRM , ECOM und HCM Beiträgen geeignet.\\
Im Bezug auf die Registrierung in jedem Forum könnte es, gerade bei neueren Foren Probleme geben. In manchen Fällen werden Registrierungsanfragen noch mit zusätzlichen Sicherheitsmechanismen wie reCaptcha geschützt. Diese sind in der bisherigen Implementierung noch nicht umgangen worden, was ein Registrieren bei diesen Webseiten unmöglich macht. Wird eine Registrierung in diesen Foren gewünscht, kann die Arbeit von Google herangezogen werden, in dem sie ihre eigenen reCaptcha automatisch von Programmen lösen lassen. Das richtig gelöste reCaptcha zusammen mit den anderen gültigen Registrierungsparametern erlaubt dann ein automatisches Registrieren in solchen Foren. Sollte der Registrierungsvorgang dennoch nicht funktionieren, können händisch ein Nutzeraccount angelegt, verifiziert und diese Daten für einen Login benutzt werden.\\
Es ist wie gezeigt möglich, sich in Internetforen anzumelden und zu registrieren, sowie Beiträge passend zu Firmenprodukten aus dem Forum zu extrahieren.
Auch ist es möglich,wenn auch mit kleinem Fehler, die Datenbankgröße des Forums durch Suchanfragen zu bestimmen sowie eine Aussage über die Relevanz eines Forums für ein Unternehmen zu generieren. Dieser Ansatz ermöglicht, das Problem der Datenknappheit für das Noise to Opportunity Produkt zu lösen, sowie rein theoretisch programmatisch Webseiten des Private Webs zu indizieren, was vorher nicht möglich war. Ein Problem besteht darin, dass die bereitgestellten Email-Adressen von der Internetseite 10minutemail \footnote{https://10minutemail.net/ checked: 10.06.2015} nicht zum Registrieren im Forum benutzt werden können. Das kann passieren, wenn zu viele Spam-Accounts mit Emails von diesem Anbieter im Forum erstellt wurden. Um das Problem zu lösen, muss entweder ein Account manuell erstellt, oder das Programm so verändert werden, dass es sich mit einem anderen Email-Provider registriert. Jedoch muss das Programm auch bei diesem neuen Provider die Möglichkeit haben, die Links in der Aktivierungsmail zu bestätigen.
Sollte bei dem Login oder Registrierungsprozess von der Webseite noch zusätzliche Parameter per Javascript eingefügt werden, wird dieses vom Programm noch nicht registriert.
Zu beachten und respektieren sind die geltenden AGB's der Internetforen, die untersucht werden sollen. Ist ein programmtechnisches Durchsuchen dieser Webseite untersagt, sollte dieses Programm nicht eingesetzt werden.