\section{Schlussfolgerung}
Das Verhältnis von Summe der gesamt klassifizierbaren Posts zu der Gesamtdatenbankgröße sollte mindestens 30 \% betragen, damit schnell verkaufsträchtige Posts gefunden werden. Hingegen, das Verhältnis von spezifischen Posts zu Gesamtdatenbankgröße, sollte nicht unter 10 \% fallen, da sonst der Aufwand und die Zeit diese Posts der Kategorie zu erfassen nicht im Verhältnis zum Gewinn stehen. So wäre in diesem Fall der Teil der Datenbank für CRM , ECOM und HCM Posts geeignet.\\
Im Bezug auf die Registrierung in jedem Forum könnte es, gerade bei neueren Foren Probleme geben. In Fällen werden Registrierungsanfragen noch mit zusätzlichen Sicherheitsmechanismen wie reCaptcha geschützt. Diese sind in der bisherigen Implementierung noch nicht umgangen worden, was ein Registrieren bei diesen Webseiten unmöglich macht. Wird eine Registirierung in diesen Foren gewünscht, kann die Arbeit von Google herangezogen weren, in dem sie ihre eigenen reCaptcha automatisch von Programmen lösen lassen. Das richtig gelöste reCaptcha zusammen mit den anderen gültigen Registrierungsparametern erlauben dann ein automatisches Registrieren in solchen Foren. Sollte der Registrierungsvorgang dennoch nicht funktionieren, kann händisch ein Nutzeraccount angelegt und verifiziert werden und diese Daten für einen Login benutzt werden.\\
Es ist gezeigt möglich, sich in Internetforen anzumelden und zu registrieren, sowie Posts passend zu Firmenpordukten aus dem Forum extrahieren.
Auch ist es möglich mit kleinem Fehler die Datenbankgröße des Forums durch Suchanfragen zu bestimmen, sowie eine Aussage über die Relevanz eines Forums für ein Unternehmen zu generieren. Dieser Ansatz ermöglicht das Problem der Datenknappheit für das Noise to Opportunity Produkt zu lösen, sowie rein theoretisch programmatisch Webseiten des Private Webs zu indizieren, was vorher nicht möglich war. Ein weiteres Problem besteht darin, dass die bereitgestellten Emailadressen von der Internetseite 10minutemail \footnote{https://10minutemail.net/ checked: 10.06.2015} nicht zum Registrieren in dem Forum benutzt werden können. Dieses kann passieren, wenn zu viele Spam-Accounts mit Emails von diesem Anbieter in dem Forum erstellt wurden. Im das Problem zu lösen, muss entweder ein Account manuell erstellt werden, oder das Programm so verändert werden, dass es sich mit einem anderen Emailprovider registriert. Jedoch muss das Programm auch bei diesem neuen Provider die Möglichkeit haben die Links in der Aktivierungsemail zu bestätigen.
Sollte bei dem Login oder Registrierungsprozess von der Webseite noch zusätzliche Parameter per Javascript eingefügt werden, wird dieses von dem Programm noch nicht registriert.
Zu beachten und respektieren sind die geltenden AGB's der Internetforen die untersucht werden sollen. Ist ein programmatisches durchsuchen dieser Webseite untersagt, sollte dieses Programm nicht eingesetzt werden.
