%\section{Schlussfolgerung}

%Der Anteil der gesamten klassifizierbaren Beiträge zur Gesamtdatenbankgröße sollte mindestens 30 \% betragen, damit schnell verkaufsträchtige Posts gefunden werden. Hingegen, sollte der Anteil von spezifischen Posts zur Gesamtdatenbankgröße nicht unter 10 \% fallen, da sonst der Aufwand und die Zeit, diese Beiträge der Kategorie zu erfassen, nicht im Verhältnis zum Gewinn stehen. So wäre in diesem Fall der Teil der Datenbank für CRM , ECOM und HCM Beiträgen geeignet.\\
%Im Bezug auf die Registrierung in jedem Forum könnte es, gerade bei neueren Foren Probleme geben. In manchen Fällen werden Registrierungsanfragen noch mit zusätzlichen Sicherheitsmechanismen wie reCaptcha geschützt. Diese sind in der bisherigen Implementierung noch nicht umgangen worden, was ein Registrieren bei diesen Webseiten unmöglich macht. Wird eine Registrierung in diesen Foren gewünscht, kann die Arbeit von Google herangezogen werden, in dem sie ihre eigenen reCaptcha automatisch von Programmen lösen lassen. Das richtig gelöste reCaptcha zusammen mit den anderen gültigen Registrierungsparametern erlaubt dann ein automatisches Registrieren in solchen Foren. Sollte der Registrierungsvorgang dennoch nicht funktionieren, können händisch ein Nutzeraccount angelegt, verifiziert und diese Daten für einen Login benutzt werden.\\
%Es ist wie gezeigt möglich, sich in Internetforen anzumelden und zu registrieren, sowie Beiträge passend zu Firmenprodukten aus dem Forum zu extrahieren.
%Auch ist es möglich,wenn auch mit kleinem Fehler, die Datenbankgröße des Forums durch Suchanfragen zu bestimmen sowie eine Aussage über die Relevanz eines Forums für ein Unternehmen zu generieren. Dieser Ansatz ermöglicht, das Problem der Datenknappheit für das Noise to Opportunity Produkt zu lösen, sowie rein theoretisch programmatisch Webseiten des Private Webs zu indizieren, was vorher nicht möglich war. Ein Problem besteht darin, dass die bereitgestellten Email-Adressen von der Internetseite 10minutemail \footnote{https://10minutemail.net/ checked: 10.06.2015} nicht zum Registrieren im Forum benutzt werden können. Das kann passieren, wenn zu viele Spam-Accounts mit Emails von diesem Anbieter im Forum erstellt wurden. Um das Problem zu lösen, muss entweder ein Account manuell erstellt, oder das Programm so verändert werden, dass es sich mit einem anderen Email-Provider registriert. Jedoch muss das Programm auch bei diesem neuen Provider die Möglichkeit haben, die Links in der Aktivierungsmail zu bestätigen.
%Sollte bei dem Login oder Registrierungsprozess von der Webseite noch zusätzliche Parameter per Javascript eingefügt werden, wird dieses vom Programm noch nicht registriert.
%Zu beachten und respektieren sind die geltenden AGB's der Internetforen, die untersucht werden sollen. Ist ein programmtechnisches Durchsuchen dieser Webseite untersagt, sollte dieses Programm nicht eingesetzt werden.

\section{Schlussfolgerung}
Unter den 100 getesteten Internetforen konnte ein erfolgreicher Registrierungsprozess bei 76 \% der nachgebauten Registrierungsseiten der Foren festgestellt werden. Bei 82 \% der Webseiten konnte sich erfolgreich eingeloggt werden. Dafür wurde manuell bei 6 Webseiten ein Nutzerkonto angelegt und validiert, sich dann aber automatisch mit den generierten Nutzerdaten eingeloggt. Bei 80 \% der Webseiten konnte die Suchform identifiziert und eine Suchanfrage korrekt abgesendet werden. Damit sollte es bei mindestens 63 \% (76 * 0.82) der Foren die im Internet verfügbar sind möglich sein, diese programmtechnisch zu durchsuchen.
Um eine Aussage darüber treffen zu können, wie relevant dieses Forum im Endeffekt für das Verkaufen eines Firmenproduktes ist, wird neben der Klassifizierung des Produktes eines Beitrags, der Kaufwunsch, der in diesem Beitrag ausgedrückt wird gespeichert. Dieser Kaufwunsch ist ein nummerischer Wert, den der Klassifizierungsservice von Berger und Hennig \cite{n2o} bei jeder Produktbestimmung eines Beitrages mit ausliefert. Für jeden extrahierten Beitrag, der einem Unternehmensprodukt zugeordnet werden kann, wird dieser Wert addiert und zum Schluss durch die Anzahl der gefundenen Produktbeiträge dividiert. Dabei ergibt sich folgende Verteilung: 

\begin{table}[h!]
\centering
\begin{tabular}{ | p{3cm} | l |}
\hline
Produkt & Kaufwunsch\\ \hline
CRM & 79 \% \\ \hline
HCM & 52 \% \\ \hline
ECOM & 54 \% \\ \hline
LVM & 57 \% \\ \hline
\end{tabular}
\caption{Errechneter Kaufbedarf für jede Produktkategorie}
\end{table}

Das bedeutet, dass fast 80 \% der CRM Beiträge in der Testdatenbank mit 22120 Beiträgen, einen Kaufbedarf äußern. Demnach drücken in der getesteten Datenbank 1816 Beiträge einen starken Kaufbedarf an einem CRM Produkt aus. Dem Verkäufer wird eine Statistik generiert und angezeigt. Er kann dann entscheiden, ob dieses Forum für den Vertrieb seines Produktes geeignet scheint. Das Verhältnis von produktspezifischen Beiträgen zu Gesamtdatenbankgröße sollte nicht unter 5 \% fallen, da sich dann dieses Forum mit anderen Themen, als dem gesuchten Unternehmensprodukt, befasst. Weiterhin sollte bei mindestens 30 \% der klassifizierten Beiträgen ein Kaufbedarf erkennbar sein, da sonst zwar produktspezifische Beiträge gefunden werden können, diese dem Verkäufer jedoch nichts bringen, da er den Autoren kein Produkt verkaufen kann. Diese Zahlen müssten mit realen Verkäufern in realen Internetforen evaluiert werden, stellen aber eine gute Richtlinie für eine Forenrelevanz dar.

\section{Ausblick}
Im Bezug auf die Registrierung in Foren könnte es, gerade bei neueren Foren, Probleme geben. In manchen Fällen werden Registrierungsanfragen noch mit zusätzlichen Sicherheitsmechanismen wie `reCaptcha` geschützt. Diese sind in der bisherigen Implementierung noch nicht umgangen worden, was ein Registrieren bei diesen Webseiten unmöglich macht. Wird eine Registrierung in diesen Foren gewünscht, kann die Arbeit von Google herangezogen werden, in der sie ihre eigenen `reCaptcha` automatisch von Programmen lösen lassen\cite{goodfellow2013multi}. Das richtig gelöste `reCaptcha`, zusammen mit den anderen gültigen Registrierungsparametern, erlaubt dann ein automatisches Registrieren in solchen Foren. Sollte der Registrierungsvorgang dennoch nicht funktionieren, kann händisch ein Nutzeraccount angelegt, verifiziert und diese Daten für einen Login benutzt werden.\\
Ein Problem besteht, wenn die bereit gestellten Email-Adressen von der Internetseite 10minutemail \footnote{https://10minutemail.net/ checked: 10.06.2015} in manchen Foren nicht zum Registrieren benutzt werden können. Das kann passieren, wenn zu viele Spam-Accounts mit Emails von diesem Anbieter im Forum erstellt wurden. Um das Problem zu lösen, muss entweder ein Account manuell erstellt, oder das Programm so verändert werden, dass es sich mit einem anderen Email-Provider registriert. Jedoch muss das Programm auch bei diesem neuen Provider die Möglichkeit haben, die Links in der Aktivierungsmail zu bestätigen.
Sollte bei dem Login oder Registrierungsprozess von der Webseite noch zusätzliche Parameter per Javascript eingefügt werden, wird dieses vom Programm noch nicht registriert.\\
In einer Testphase sollten Datenbankgrößen ausgerechnet werden um abzuschätzen ob die Formel zur Berechnung der Datenbankgröße, die Größen richtig berechnet. Sollte dieses nicht der Fall sein kann eine andere Formel und zusätzlich eine Gewichtung der Suchwörter nach dem Zipf'schen Gesetz \cite{leopold2002zipfsche} der jeweiligen Sprache, implementiert und getestet werden\cite{jiang2009selectivity}.
Zu beachten und respektieren sind die geltenden AGB's der Internetforen, die untersucht werden sollen. \textbf{Ist ein programmtechnisches Durchsuchen dieser Webseite untersagt, sollte dieses Programm nicht eingesetzt werden!}

