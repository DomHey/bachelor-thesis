\section{Evaluation}
\subsection{Bestimmung der Forenrelevanz}

Es geht darum zu evaluieren, ob eine Vorhersage darüber getroffen werden kann, wie groß eine Forendatenbank und wie relevant dieses Forum für ein jeweiliges Unternehmen ist.

Es wird zunächst die Datenbankgröße bestimmt wie in Kapitel [] beschrieben. Die Testdatenbank wurde in 3 Segemente geteilt. Segement 1 enthält nur Beiträge aus Forum 1, Segement 2 nur aus Forum 2 und Segement 3 nur Beiträge aus Forum 3. 
Segement 1 und 2 beinhalten englische Beiträge, wohingegen Segement 3 meist deutsche Beiträge enthält. Die Postanzahl aller 3 Segemente ist bekannt. Um eine realistische Einschätzung zu erhalten, wie zuverlässig die Vorhersage der Gesamtdatenbankgröße ist, wurde der Test für jedes Segement 3 mal wiederholt. Ein Testdurchlauf besteht aus 3 mal 500 zufälligen Wörtern, die an die Datenbank in der passenden Sprache gesendet werden. Die ermittelten Datenbankgrößen nach 500 Wörtern werden addiert, durch die Gesamtanzahl der Testläufe dividiert und danach mit der bekannten Gesamtanzahl der Posts in dem Datenbanksegment verglichen.

\begin{table}[h!]
\begin{tabular}{ | p{3cm} | p{3cm} | p{3cm}| p{3cm} |} \hline
Durchlauf & Anzahl Dok in S1 & Anzahl Dok in S2 & Anzahl Dok in S3 \\ \hline
1 & 12572 & 18374 & 8079 \\ \hline
2 & 10400 & 21155 & 7896 \\ \hline
3 & 15017 & 20952 & 8343 \\ \hline
Arith. Mittel & 12663 & 20160 & 8016 \\ \hline
org. Größe & 12133 & 21221 & 8184 \\ \hline
Fehler & +4\% & -5\% & -1\% \\ \hline
\end{tabular}
\caption{Ermittelung der Datenbankgrößen mit zufälligen Wörtern}
\end{table}

Der Fehler beträgt maximal 5\% . Dieses bestätigten die zwei weiteren Testdurchläufe. Damit ist eine Voraussage einer Forendatenbank bis auf 10\% Differenz genau möglich.

Im nächsten Schritt sollen die einzelnen Produkte im Forum gesucht werden. Dazu werden, wie in Kapitel[] beschrieben, beschreibende Wörter für jedes einzelne Produkt generiert. Diese Keywords werden gesucht und die Datenbankgröße aufgrund des Verhältnisses zwischen überlappenden und einzigartigen Dokumenten, wie in Kapitel[] beschrieben, berechnet. Die zu Testzwecken verwendete Datenbank enthält 21220 Beiträge.

\begin{table}[h!]
\centering 
\begin{tabular}{ | p{3cm} | l |}
\hline
Produkt & Anzahl der Dokumente in DB \\ \hline
CRM & 2386 \\ \hline
HCM & 3138 \\ \hline
ECOM & 3641 \\ \hline
LVM & 1769 \\ \hline
Gesamt & 10934 \\ \hline
\end{tabular}
\caption{Anzahl der Dokumente je Kategorie in der Testdatenbank. Der Rest der Beiträge konnte keinem Produkt eindeutig zugeordnet werden.}
\end{table}

\newpage

\begin{table}[h!]
\begin{tabular}{ | p{3cm} | l | l |}
\hline
Produkt & Berechnete Anzahl der Dokumente in DB & Fehler\\ \hline
CRM & 3800 & +59 \%\\ \hline
HCM & 9668 & +338 \% \\ \hline
ECOM & 10604 & +265 \%\\ \hline
LVM & 39541 & +2235 \%\\ \hline
\end{tabular}
\caption{Errechnete Datenbankgröße nach Produkt mit Suchworten nach tf-idf Maß}
\end{table}

Zu verzeichnen ist, dass vom Programm die Größe der einzelnen Produktkategorien maßlos überschätzt wurde. Deshalb wird zur Ergebnisverbesserung ein zusätzlicher Schritt eingefügt. Es werden wie bisher die Keywords zu den jeweiligen Produkten im Forum gesucht, jedoch bevor der Post zur Berechnung des Gesamtkorpus beiträgt, durch einen Klassifizierungsservice \cite{n2o} analysiert. Dieser bestimmt aus einem gegebenen Text auf Grundlage der selben Broschüren die zur Berechnung der tf-idf- Keywords verwendet werden, welchem Produkt der Beiträge am ehesten entspricht.

\begin{figure}[h!]
\begin{lstlisting}[language=HTML5]
http://192.168.42.54:9001/predictions?text=Hi, I want to by an CRM product
\end{lstlisting}
\caption{Senden eines Testtextes an den Analyseservice}
\end{figure}

\newpage

\begin{figure}[h!]
\begin{lstlisting}[language=HTML5]
{
"product": [
{
"product": "CRM",
"prob": 0.9999979024787202
},
{
"product": "ECOM",
"prob": 0.0000020975212798552666
},
{
"product": "LVM",
"prob": 3.0982005787342066e-25
},
{
"product": "None",
"prob": 2.2462124385020632e-66
},
{
"product": "HCM",
"prob": 1.5018464896178129e-218
}
]
}
\end{lstlisting}
\caption{Antwort des Analyseservices mit entsprechender Klassifizierung des Textes}
\end{figure}

Der Beitragstext (Abbildung 18) würde als CRM Beitrag richtig eingestuft werden. Wenn gerade die CRM Forendatenbankgröße berechnet werden sollte, wird dieser Beitrag mit in die Berechnung einfließen. Alle Beiträge die als nicht CRM Beiträge eingestuft werden, werden in dieser spezifischen Ermittlung nicht betrachtet.\\
Mit maximal 100 generierten Keywords, gewichtet nach der höchsten tf-Idf, werden folgende Resultate erreicht:

\begin{table}[h!]
\begin{tabular}{ | p{3cm} | l | l |}
\hline
Produkt & Berechnete Anzahl der Dokumente in DB & Fehler \\ \hline
CRM & 2270 & -5\% \\ \hline
HCM & 2730 & -14 \% \\ \hline
ECOM & 4388 & +17\% \\ \hline
LVM & 1670 & -5\% \\ \hline
\end{tabular}
\caption{Errechnete Datenbankgröße je Produkt mit sortierten tf-idf Wörtern und Analyseservice}
\end{table}

Die anderen 2 Testsegmente der Datenbank konnten mit ähnlich genauen Fehlern überprüft werden.
Ein Fehler von maximal 20\% mit nur maximal 100 Suchanfragen ist eine zufriedenstellende Größe. Aus der Datenbankgröße und den spezifischen Produktdatenbankgrößen lässt sich bestimmen, ob das Forum zum Verkauf eines Produktes geeignet ist oder nicht.
