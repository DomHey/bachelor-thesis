\section{Ermitteln der Größe der Forendatenbank}
\subsection{Intro}
Die Ermittelung der Datenbankgröße wird bei dem späteren treffen der Aussage über die Forenrelevanz für ein Produkt wichtig.
Es ist wichtig die annähernde Anzahl der Posts in dem Forum zu kennen, um bei einer Langzeitanalyse festzustellen, wie schnell das Forum wächst, bzw. wie viele Einträge pro Tag hinzu kommen. Daraus kann ermittelt werden, mit welcher Wahrscheinlichkeit neue Beiträge zu einem Verkaufsprodukt mit welcher Frequenz eingestellt werden.\\
\subsection{Ermitteln des Suchfeldes}
Um die Datenbank analysieren zu können muss zuächst das Suchfeld, welches es in den meisten Foren gibt, um die vorhandenen Posts zu druchsuchen, gefunden werden.
Dazu wird das HTML das nach dem erfolgreichen Einloggen auf weitere Form-Felder durchsucht. Die gefundenen Felder werden klassifizeirt indem sich die Namen-Attribute der Input-Felder dieser Form angeschaut werden. Sollte dieser Name aus nur einem Buchstaben bestehen, vorzugsweise ein `q`, da das bei den meisten Webseiten für den Paramater query steht, werden Puntke vergeben.
Danach werden alle Attribute der Input-Felder der Form durchsucht. Sollten in den Attributen die Zeichenketten `search` oder `suche` vorkommen werden weitere Punkte vergeben.\\
Zum Schluss ist die Form mit dem höchsten Punktwert, die Form die am wahrscheinlichsten die Suchbar als Input-Feld enthält.
Nun wird der Pfad der Suche aus der Form extrahiert, indem das `action` Attribut der Form ausgelesen wird. Dieser Wert wird an die Toplevel Domain angefügt und stellt die Suchurl dar. Der Queryparameter wird aus dem Name-Attribut des Input-Feldes extrahiert und an die Suchurl angehangen. Sollte zum Beispiel das Name-Attribut des Input-Feldes ein `q` sein würde sich die vorläufige Suchurl so zusammensetzen: `Toplevel Url` + `?q=`. An den Queryparameter q werden im folgenden die einzelnen Suchworte, die zur Ermittelung der Datenbankgröße beitragen angehängt und an den Server abgesendet.\\
Diese Wörter, die die Datenbank bestimmen sollen werden aus einer Datei gelesen, die die am häufigsten benutzen Wörter in einer Sprache sortiert auflistet. Aus diesen Wörtern werden zufällig 500 Wörter gezogen ,in die Url eingefügt und abgesendet.\\
\subsection{Berechnung der Datenbankgröße}
Nach jedem abgesendeten Request wird der zurückgegebene Inhalt analysiert. Dieser enthält die Posts, die von der 
forumseigenen Suchmaschine zurückgegeben werden, wenn man nach einem spezifischen Wort sucht. In dem Inhalt werden nun alle Hyperlinks gesucht und gespeichert. Diese Prozedur wird 500 mal wiederholt. Zum Schluss werden die Gesamtzahl an gewonnenden Links berechnet, sowie die Anzahl der Links, die bei unterschiedlichen Suchwörtern identisch sind. Links die in mehr als 20\% der Gesamtanfragen vorkommen werden ignoriert, da es sich meist hierbei dann um statische Links die auf der Suchseite befinden handeln. Diese Links zeigen dann nicht auf Posts die aus der Suchanfrage generiert werden, sondern auf statische Elemente der HTML-Webseite wie Navigationsbar oder ähnliches. Diese würden dann die Berechnung der Datenbankgröße verfälschen und werden deshalb ignoriert.
DIe finale Größe der Datenbank lässt sich mit hilfe der Formel \(\frac{u}{1-OR^{-2.1}}\) berechnen. Hierbei ist u die Anzahl der einzigartigen Dokumente und OR das Verältnis von allen Dokumenten zu doppelten[aus dem paper ]\\
Dieser ganze Vorgang wird drei mal wiederholt und aus den drei resultierenden Datenbankgrößen das arithmetische Mittel gebildet. Diese Zahl bestimmt, vie viele Posts sich insgesamt in der Forendatenbank befinden.

\newpage