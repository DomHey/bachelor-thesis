\section {Intro}
In dem Bachelorprojekt `Noise to Opportunity` wurde eine Softwarelösung entwickelt, die es ermöglicht, aus den unmengen von Social Media Posts, aus Linkedin, Facebook, Xing und Twitter diejenigen extrahiert, die einen Kaufwunsch an einem bestimmten Produkt äußern. Innerhalb eines Jahres sind ca. 25.000 Posts zusammengekommen, von denen allerdings nur 14.000 Beiträge einen Kaufbedarf ausgedrückt haben. Von diesen 14.000 Beiträgen konnten nur ca 1.000 mit einer Wahrscheinlichkeit von über 60 \% zu einem zu vertreibenen Produkt zugeordnet werden. Dieses bedeutet das nur 4 \% der Beiträge überhaupt eine Verkaufschance bieten.
Es ist zu sehen, dass es zu wenig sinvolle Daten gibt um dieses Produkt in Großfirmen einzusetzen. Dieses lässt sich beheben, indem die Datenaquise in mehr Internetforen / Portalen fortgesetzt wird.\\
Diese neuen Datenquellen müssen durchsucht werden. Es muss per Hand ein Crawler für die Webseite geschrieben werden, wenn sie keine API anbietet. Dazu muss die Webseite aufwendig analysiert werden um die ganzen Login- und Suchanfragen in diesem Forum zu verstehen. Ist dieses geschafft, stellt man fest das dieses Forum keine sinnvollen Daten bereitstellt und somit ist die ganze Zeit, die investiert wurde, vergebens.\\
Aus diesem Grund wird hier eine generische Lösung beschreiben und getestet, wie das händische Analysieren und Crawlen einer Webseite automatisiert werden kann.Ziel istes, einerseits der Webseite ein unnötiges crawlen zu ersparen, wenn keine passenden Daten in dem Forum vorhanden sind und anderseits eine generelle Aussage über die Relevanz, das heißt Forengröße und am wahrscheinlichst zu verkaufenden Produkt, treffen zu können. Anhand dieser Informationen kann entschieden werden, ob es sich lohnt einen Crawler manuell zugeschnitten auf die Webseite zu erarbeiten oder nicht.\\
Dehalb beschäftig sich diese Arbeit mit der These, dass es möglich sein sollte, sich automatisch in Internetforen zu registrieren, anzumelden und automatisiert die Datenbankgröße des Forums zu bestimmen, sowie eine Aussage darüber zu treffen, wie geeignet das Forum zum verkaufen eines bestimmten Produktes ist. Die theoretischen Details für die spätere Umsetztung werden in den nachfolgenden Seiten beschrieben, sowie die Umsetzung in dem Kapitel `Evaluation` evaluiert.
\newpage