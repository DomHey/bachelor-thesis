\section{Hintergrund}

Im Rahmen des Bachelorprojektes am Hasso-Plattner-Institut wurde innerhalb eines Jahres die Softwarelösung Noise to Opportunity, in einer Gruppe von 8 Personen, entwickelt. Ich habe mich mit der Datenakquise beschäftigt. Dazu habe ich eine Vielzahl von Webseiten analysiert um die Registrierungs-, Einlog- und Suchprozesse zu verstehen und programmatisch nachzubauen. Bei dieser Tätigkeit ist aufgefallen, dass die meisten Internetforen eine ähnliche Struktur besitzen. Anfänglich war die Herausforderung, sich mit einem händisch angelegten Nutzeraccount, in dem Forum automatisch anzumelden und das Forum zu durchsuchen. Als dieses, durch Analyse der Webseiten, sich wiederholende Prozesse erkennen ließ, entstand die Idee, sich auch automatisch registrieren zu wollen.
Viele der Registrierungsformen besitzen gleiche Attributnamen, die sie in dem HTML-Quellcode identifizieren. Der Registrierungsprozess folgt dem gleichen Schema und lässt sich demnach genau wie das Einloggen und das Suchen automatisieren. Es entstand die Idee ein Programm zu entwickeln, welches automatisch Internetinhalte hinter POST und GET Formularen indizieren kann.

\subsection{Related Work}

Jianuguo Lu \cite{lu2008efficient} entwickelt in seiner Arbeit eine Formel, die eine Bestimmung der Datenbankgröße ermöglicht, wenn die Datenbank nur über ein Suchanfragen Interface angefragt werden kann. Er beschreibt die Abhängigkeit von der Anzahl der Suchanfragen zu Anzahl der Einträge, die auf eine Suchanfrage zurückgegeben werden. Getestet wird an Datenbanken mit bekannten Größen.Dabei wird eine Konvergenz zur realen Datenbankgröße ab 200 Suchanfragen festgestellt. Die Formel aus dem Heterogenen Model (Seite 1) stellt die Grundlage für meine Errechnung der Forendatenbank Größe, sowie für die Errechnung der Datenbankgröße für ein spezifisches Firmenprodukt.


