\section{Hintergrund}

Im Rahmen des Bachelorprojektes am Hasso-Plattner-Institut wurde innerhalb eines Jahres die Softwarelösung Noise to Opportunity, in einer Gruppe von 8 Personen, entwickelt. Ich habe mich mit der Datenakquise beschäftigt. Dazu habe ich eine vielzahl von Webseiten analysiert um die Registrierungs-, Einlog- und Suchprozesse zu verstehen und programmatisch nachzubauen. Bei dieser Tätigkeit ist aufgefallen, dass die meisten Internetforen eine ähnliche Struktur besitzen. Anfänglich war die Herausvorderung, sich mit einem händisch angelegten Nutzeraccount, in dem Forum automatisch anzumelden und das Forum zu durchsuchen.  Als dieses, durch Anaylse der Webseiten, sich wiederholende Prozesse erkennen ließ, entstand die Idee, sich auch automatisch registrieren zu wollen.
Viele der Registrierungsformen besitzen gleiche Attributnamen, die sie in dem HTML-Quellcode identifizieren. Der Registrierungsprozess folgt dem gleichen Schema und lässt sich demnach genau wie das Einloggen und das Suchen automatisieren. Es entstand die Idee ein Programm zu entwickeln, welches automatisch Internetinhalte hinter POST und GET Forumlaren indizieren kann.

\subsection{Related Work}